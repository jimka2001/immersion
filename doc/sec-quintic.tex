\section{Quintic: degree=5}
\label{sec.quintic}

A quintic polynomial has the form $P(x) = a x^5 + b x^4 + c x^3 + d x^2 + e x + f$.
If $a=0$ than $P(x)$ is really
a quartic polynomial (degree 4) and can be solved using the techniques described in Section~\ref{sec.quartic}.
Since a quintic polynomial has odd degree (if $a\neq 0$),
then, like the cubic (Section~\ref{sec.cubic}), it is guaranteed to have a real root.  This fact
is guaranteed, because if
$a>0$ then $P(x)\to \infty$ for $x>>0$ and $P(x)\to -\infty$ for $x<<0$.

Given this fact, we can find a root, $r$ using a binary search as we did for the cubic.
Since $P(0) = f$, then we can consider two cases.
\begin{enumerate}
\item If $f>0$ then search for a root on the negative x-axis
\item If $f<0$ then search for a root on the positive x-axis
\end{enumerate}

Once the root $r$ is found, factor out $(x-r)$ from $P(x)$ to result in a quartic, which we can solve using
the technique in Section~\ref{sec.quartic}.
\begin{align*}
  P(x) &= a x^5 + b x^4 + c x^3 + d x^2 + e x + f\\
  &= (x - r) (Ax^3 + Bx^2 + Cx + D)
\end{align*}

Where

\begin{align*}
  A &= a\\
  B &= b + a r\\
   &= b + A r\\
  C &= c + b r + a r^2\\
  &= c + (b + a r)r\\
  &= c + B r\\
  D &= d + c r + b r^2 + a r^3\\
  &= d + ( c + b r + a r^2)r\\
  &= d + C r\\
  E &= e + d r + c r^2 + b r^3 + a r^4\\
  &= e + ( d + c r + b r^2 + a r^3)r\\
  &= e + D r
\end{align*}


Having determined $A$, $B$, $C$, $D$, and $E$, the roots of $(Ax^3 + Bx^2 + Cx + D)$ can be
found with a call to \code{find\_quartic\_roots}, which you implemented in Section~\ref{sec.quartic}.

This exercise is left as an exercise for the student.
