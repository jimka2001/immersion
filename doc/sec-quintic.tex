\section{Quintic: degree=5}
\label{sec.quintic}

A quintic polynomial has the form $P(x) = a x^5 + b x^4 + c x^3 + d x^2 + e x + f$.
If $a=0$ than $P(x)$ is really
a quartic polynomial (degree 4) and can be solved using the techniques described in Section~\ref{sec.quartic}.
Since a quintic polynomial has odd degree (if $a\neq 0$),
then, like the cubic (Section~\ref{sec.cubic}), it is guaranteed to have a real root.  This fact
is guaranteed, because if
$a>0$ then $P(x)\to \infty$ for $x>>0$ and $P(x)\to -\infty$ for $x<<0$.

Given this fact, we can find a root, $r_5$ using a binary search as we did for the cubic;
factor out $(x-r_5)$ from $P(x)$ to result in a quartic, which we can solve using
the technique in Section~\ref{sec.quartic}.

This exercise is left as an exercise for the student.
