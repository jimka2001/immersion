\section{Quintic: degree=5}
\label{sec.quintic}

This exercise is left as an exercise for the student.  However, the
following are some notes to help lead the student in the right
direction.

\begin{figure}
\centering
\input{pfg-quintic}
\caption{Quintic}
\label{fig.quintic}
\end{figure}

\begin{align*}
  P(x) &= x^5 - \frac{1}{2} x^4 + 5 x^3 + \frac{5}{2} x^2 + 4 x - 2
\end{align*}


A quintic polynomial has the form $P(x) = a x^5 + b x^4 + c x^3 + d x^2 + e x + f$.
If $a=0$ then $P(x)$ is really a quartic polynomial
(degree~4) and can be solved using the techniques described in
\S~\ref{sec.quartic}.  Since a quintic polynomial has odd degree
(if $a\neq 0$), then, like the cubic (\S~\ref{sec.cubic}), it is
guaranteed to have a real root.  This fact is guaranteed, because if
$a>0$ then $P(x)\to \infty$ for $x>>0$ and $P(x)\to -\infty$ for
$x<<0$.

Given this fact, we can find a root, $r$ using a binary search as we
did for the cubic.  Since $P(0) = f$, then we can consider two cases.
\begin{enumerate}
\item If $f>0$ then search for a root on the negative x-axis
\item If $f<0$ then search for a root on the positive x-axis
\end{enumerate}

Once the root $r$ is found, factor out $(x-r)$ from $P(x)$ to result
in a quartic, which we can solve using the technique in
\S~\ref{sec.quartic}.
\begin{align*}
  P(x) &= a x^5 + b x^4 + c x^3 + d x^2 + e x + f\\
  &= (x - r) (Ax^4 + B x^3 + C x^2 + D x + E)
\end{align*}

Where

\begin{align}
  A &= a\label{eq.8.A}\\
  B &= b + a r\nonumber\\
   &= b + A r\label{eq.8.B}\\
  C &= c + b r + a r^2\nonumber\\
  &= c + (b + a r)r\nonumber\\
  &= c + B r\label{eq.8.C}\\
  D &= d + c r + b r^2 + a r^3\nonumber\\
  &= d + ( c + b r + a r^2)r\nonumber\\
  &= d + C r\label{eq.8.D}\\
  E &= e + d r + c r^2 + b r^3 + a r^4\nonumber\\
  &= e + ( d + c r + b r^2 + a r^3)r\nonumber\\
  &= e + D r\label{eq.8.E}
\end{align}


Having determined $A$, $B$, $C$, $D$, and $E$, the roots of
\[A x^4 + B x^3 + C x^2 + D x + E\] can be
found with a call to \code{find\_quartic\_roots}, which you
implemented in \S~\ref{sec.quartic}.



% LocalWords:  Quintic quintic quartic
