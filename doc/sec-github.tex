\section{GitHub}
\label{sec.github}
In this course we will be experimenting with software together.  You
will be able to share your code with your classmates, and with the
world.  However, you control when you want to make your code available
to the world, and when you want to incorporate your classmate's code
into your development area.


\begin{figure}[h]
  \centering
  \includegraphics[width=0.9\textwidth]{GitHub-How-to-use-GitHub-Edureka-300x241.png}
%%  \url{https://www.edureka.co/blog/how-to-use-github/}
  \caption{GitHub is a cloud service that hosts code repositories.
    You connect to GitHub either by web browser or directly from
    \emph{git}.}
\end{figure}




To achieve this goal we have a central repository of code, hosted by
GitHub on the cloud.  You will download (clone) a copy of this
repository, edit existing code, create new code, and then
occasionally upload your contributions back to the central repository,
making your contributions available to everyone else.

GitHub is a web based service which hosts thousands of projects, small
and large, in the cloud.  Typically, users have a development
environment on their local computer (laptop or desktop) and
periodically upload and download changes to the cloud.

The URL \url{https://github.com/jimka2001/mgs-2024}, is the public
entry point to the GitHub project which we will work on in this
MGS-2024 course.  

You must create an account on GitHub (if you don't have one already).
This will be done in Section~\ref{sec.setup}.


\subsection{Getting Started}
\label{sec.setup}

I recommend you use the Chrome web browser.

\includegraphics[width=0.8\textwidth]{chrome.png}.

We need to set up the development environment.  Before you can learn
to use a piece of software, you have to install it. These steps should
help.

\subsection{Account Creation}
  
Create a GitHub account using an abstract user name.  Don't use
your real name.  This is self explanatory at the URL
\url{https://github.com/join}.


\noindent\includegraphics[width=0.8\textwidth]{github-join.png}



\subsection{Open the Repository}
  
Open \url{https://github.com/jimka2001/immersion} with your web browser.

\noindent\includegraphics[width=0.95\textwidth]{github-immersion.png}




\subsection{Fork yourself a copy}

Fork the repository.  This gives you a private copy.  You can make changes
  here without disturbing anyone.  

\noindent \includegraphics[width=\textwidth]{github-fork-repo.png}



\subsection{Open a Python File}
  
\begin{itemize}
\item Navigate to \code{src/hello} to see something like this.

\noindent\includegraphics[width=0.9\textwidth]{find-hello.png}



\item Click \code{hello.py} to open the file in an editor pane.  You
  should see something like what is shown here:

\noindent\includegraphics[width=0.9\textwidth]{hello-function.png}


\item Open the file using \code{github.dev}

\noindent\includegraphics[width=0.9\textwidth]{github-dev.png}


\item Wait while setting up.

\noindent\includegraphics[width=\textwidth]{github-dev-setup.png}

\item If this setup seems to never finish, it may mean you need to use Chrome as your web browser.



\noindent\includegraphics[width=\textwidth]{github-dev-opened.png}

\item Finally, the GitHub development environment should be opened.

\end{itemize}

\subsection{Set up the Editor}
  
\begin{itemize}

\item Now you can edit your code but you cannot run nor test it.

\item Find the icon \includegraphics[width=2cm]{run-debug.png} on the left hand side of the browser window.   Press that icon to see the following message.

\noindent\includegraphics[width=0.8\textwidth]{continue.png}

\item Press continue, and you should see a prompt such as the following to create a code space.   

\noindent\includegraphics[width=0.9\textwidth]{create-code-space.png}

\item Select \includegraphics[width=8cm]{select-create.png}.

\item You may be asked how many cores do you want.

\noindent\includegraphics[width=0.9\textwidth]{select-cores.png}

\item You should select the minimum: \includegraphics[width=7cm]{two-cores.png}.

\item You'll probably now need to reopen the \code{hello.py} file.

\noindent\includegraphics[width=0.9\textwidth]{re-open-python-file.png}

\item You may be asked to install the Python extension.  Press \textbf{Install}.

\noindent\includegraphics[width=0.9\textwidth]{install-python-extension.png}


\item When it finishes installing, you should see something like this:

\noindent\includegraphics[width=0.9\textwidth]{python-extension.png}

\item Reopen the explorer: \includegraphics[width=7cm]{explorer2.png}, and select the \code{hello.py} file.

\noindent\includegraphics[width=0.5\textwidth]{explorer.png}
\end{itemize}

\subsection{Make a Sample Run}

Find the icon \includegraphics[height=1.0cm]{run-triangle.png}
  in the top-right of the editor window.  Click the triangle, to see a sample run/execution of the code.

\noindent\includegraphics[width=\textwidth]{hello-terminal-output.png}


\subsection{Challenges for the Student}

Understanding errors and debugging is difficult, but it is part of programming.

Experiment with the simple pieces of code in the above sections.
Insert spaces and press the run button.  Look at the error messages
produced. Remove some quotation marks or parentheses (leaving
unbalanced quotations marks or parentheses)---again look at what error
messages you see when you try to run invalid code.


\begin{itemize}
\item remove and add some spaces at the beginning of a line
\item change the indentation
\item unbalance the parentheses
\item unbalance the quotation marks
\item put extra spaces inside the quotation marks
\item change the name of the \code{hello} function at definition site or call site.
\item figure out how to undo your changes in the test editor to make the code work again.
\end{itemize}

Question:  What is the difference between a synax error and a logical error?
\clearpage

